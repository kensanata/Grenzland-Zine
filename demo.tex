\documentclass[11pt]{wbzine}
%packages
\usepackage{lipsum}
\usepackage[utf8]{inputenc}
\usepackage[T1]{fontenc}
\usepackage[ngerman]{babel}
\usepackage{coelacanth}
%\usepackage{imfellEnglish}
%\renewcommand*\sfdefault{ugq}


\title{Der Krähenfuß}
\date{Jahrgang 1, Ausgabe 1, \today}

\begin{document}
\maketitle
\tableofcontents

\begin{titlepage}
\centering
{\bfseries\fontsize{70}{55}\selectfont Würfel}

\hrulefill

Vol. 1, Issue 1, Oktober 2020

	  \includegraphics[width=\textwidth]{coverart.png}

{\Huge The Zine Class \par}%

\end{titlepage}

\tableofcontents

\begin{multicols}{2}

\section{Ein Abschnitt}
\by{Wanderer Bill}

\lipsum

\subsection{Ein Unterabschnitt}

\smallmap{townmap.png}

\lipsum

\begin{tcolorbox}
Facilisi nonummy euismod, magna ea ut iusto. Laoreet ex in iusto nibh nulla nostrud iusto wisi nonummy suscipit tation quis nonummy dignissim?
\end{tcolorbox}

\lipsum

\begin{tabularx}{\columnwidth}{cZ}
1 & 3 Goblins \\
2 & 1 Troll \\
3 & eine andere Abenteurergruppe \\
4 & seltsame Geräusche \\
5 & 3W10 Ratten \\
6 & nur das Geräusch von Wassertropfen \\
\end{tabularx}


\lipsum

\begin{tcolorbox}
Facilisi nonummy euismod, magna ea ut iusto. Laoreet ex in iusto nibh nulla nostrud iusto wisi nonummy suscipit tation quis nonummy dignissim?
\end{tcolorbox}

\section{Ein Abschnitt in dem eine große Tabelle vorkommt}
\by{John Doe}

\lipsum

\end{multicols}
\begin{tabularx}{\textwidth}{cZ}
A & Here is an mysterious altar. In the corners of the room are
plenty of spiderwebs \\
B & A few puddles with dark murky water \\
\end{tabularx}
\begin{multicols}{2}

\lipsum

\bigmap{dungeonmap.png}

\lipsum


\end{multicols}
\end{document}
