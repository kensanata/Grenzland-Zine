\section{Dungeon: Zisterne der Auferstehung}

\subsection{Kurzfassung}

Lange Zeit war der Artemistempel, nördlich von Quastenholm, verlassen. Nur die
Zisterne wurde von den Menschen noch genutzt. Vor drei Jahren bahnte sich Balto
einen Weg aus den Tiefen unter dem Tempel an die Oberfläche. Entflohen aus
seiner Gefangenschaft, versucht er zu verhindern das seine Unterdrücker jemals
an die Oberfläche kommen. Dazu hat er eine kleine Gefolgschaft aus
Werratten um sich gescharrt. Aus den Tiefen hat er auch eine Faustgroße
gallertartige Masse mitgebracht und diese im, mit Wasser gefüllten,
Zisternebecken ausgesetzt um ihre Entwicklung zu beobachten.

Vor zwei Jahren trafen Alwin, der die vergrabene Siedlung nach Wertvollem
absuchte, und Balto eine Abmachung. Alwin hält andere ab in die vergrabene
Siedlung zu gehen und Balto übergibt ihm monatlich Schätze, die er dort
gefunden hat. Die Abmachung konnte nicht lange von Alwin gehalten werden, da
nach drei Monaten bereits Echsenmenschen nach etwas in der Siedlung suchten. Also
traf Balto mit den Echsenmenschen eine Abmachung und überließ diesen die
Siedlung wenn sie andere davon abhalten unter die Zisterne zu kommen.

Alwin hat durch Zufall herausgefunden das die Gallertmasse in der
Zisterne lange verstorbene Menschen wieder zurückbringen kann und
nutz dies aus, um einen Kult um die Zisterne zu erschaffen. Durch
einen weiteren Zufall ist eine Werratte in seinen Bann geraten, mit
der er sich weitere Werratten schafft, um Balto von dort unten
vertreiben zu können, damit er selber wieder nach den Schätzen suchen
kann. Alwin verfolgt nur dieses Ziel, und erkennt nicht die
anderen Möglichkeiten die sich ihm bieten.

\subsection{Gerüchte}

\begin{tabularx}{\columnwidth}{cZ}
1 & Das Trinkwasser in und um Quastenholm ist versalzen. (wahr) \\
2 & Echsenmenschen versuchen Quastenholm zu übernehmen. (falsch) \\
3 & Knecht … hat in Bauer Bertrams Jauchegrube Gold im Wert eines
    Jahreslohns gefunden (wahr, Eigang zu Ebene 2) \\
4 & Längst Verstorbene werden hier von den Toten zurückgebracht.
    Dafür braucht es nichts weiter als die Erinnerung an die Person.
    (falsch, durch das Wesen in der Zisterne werden aus Erinnerungen Kopien 
    erschaffen: Gallertmenschen). \\
\end{tabularx}

\subsection{Raumbeschreibungen}

Räume 1 - 7 haben gemauerte Wände und steinerne Bodenplatten. Alle
anderen ``Räume'' wurden von den Werratten mit den Händen gegraben
und bestehen aus Erde und Steinen. Die Wände und Böden sind weder
glatt noch eben. Die Werratten in menschlicher Form haben lange,
dünne Schnautzbärte (Männer und Frauen).

Nördlich von Quastenholm steht ein alter, nicht mehr genutzter Artemis Tempel.
Über die Dachrinnen des Tempels wird Regenwasser in das Zisternenbecken
geleitet. In 3~m Höhe sind 0,5~m x 0,5~m große Fensteröffung im Abstand von 3~m
angeordnet um den Innenräumen Licht zu spenden. Die Tür des Tempels liegt vor
dem Eingang. Südlich des Tempels steht ein Brunnen der direkt in das
Zisternenbecken (Raum 4) führt.

\subsection{Ebene 0}

\subsubsection{1 - Zeremonienraum:} Jeweils an der Nord- und an der Südwand
stehen drei Statuen die Artemis in einer Jagdpose zeigen. In der Mitte des
Raumes liegen aufgehäufte Holz- und Schuttreste. Die Tür zum Nachbarraum ist
angelehnt. Unter der südwestlichten Statue ein Durchbruch zu G1. In der Nähe
dieser Statue befinden sich Schleifspuren auf den Bodenplatten.

\subsubsection{2 - Hinterraum:} Entlang der Wände haben Ratten aus Unrat und
Holzresten ihre Nester gebaut. In dem Raum befinden 36 Ratten (6 Rudel), die
sich aber nur aggressiv Verhalten wenn man sich ihnen oder ihren Nestern
nähert. Im Nest am südöstlichsten Punkt im Raum steckt ein mit Feueroplaen
besetzter goldener Gürtel (500~GM). Eine Wendeltreppe führt nach unten in den
Keller.

\textbf{Rattenrudel} 6 Ratten mit jeweils 1~TP RK~9~[10] 1W6 + Krankheit NM
BW~6'' (schwimmend 3'') ML~5 EP~10 

Krankheit: 1 in 20 Chance auf Infektion. Rettungswurf
gegen Gift. 1 in 4 Chance Tod in 1W6 Tagen, ansonsten ist das Ziel krank
und bettlägerig für einen Monat.

\subsection{Ebene 1}

\subsubsection{3 - Keller:} Die Wendeltreppe führt hinauf in den Hinterraum
(Ebene 0). Südlich der Wendeltreppe lauert eine riesige Schwarze Witwe an der 3
Meter hohen Decke.  In der Südostecke des Raumes hängt ein Spinnenetz. Darunter
liegen meschliche Knochen und 65~GM. An der östlichen Wand ist ein Durchbruch
zu Raum 12.

\textbf{Riesige Schwarze Witwe} TW~3* RK~6~[13] 2w6 + Gift K2 BW~6'' (12''
im Netz) ML~8 EP~300 
Gift: tötet Opfer innerhalb einer Phase (Rettungswurf Gift)

\subsubsection{4 - Zisterne:} Ein Steg führt um das Zisternenbecken. An allen
vier Wänden hängen große Wandteppiche. Hinter dem westlichen befindet sich der
Zugang zu Raum 7. Vier schmucklose Säulen stützen die Decke. Ein Wanddurchbruch
verbindet die Zisterne mit Raum 9.

\bigmap{img/Zisterne1und0.png}

Das Zisternenbecken ist gefüllt mit einer gallertartigen Masse die sich von
Wasser ernährt. Sie ist durchsichtig. Wer sie berührt muss einen Rettungswurf
gegen Tod durchführen. Bei einem Fehlschlag stirbt der Berührende sofort an
einem Herzstillstand. Die Masse entzieht dem Opfer über 1W6 Tage das
Körperwasser und entsorgt den wasserlosen Rest im Loch in der Mitte des
Zisternenbeckens. Bei Erfolg erschafft die Masse ein Ebenbild (Gallertmensch)
der am meisten vermissten Person innerhalb der nächsten 2W6 Tage.

Der Gallertmensch entspricht sowohl den wahren als auch falschen Erinnerungen
des Berührenden.  Die Masse selber erleidet keinen Schaden durch Waffen oder
Magie und kann auch nicht vergiftet, versteinert oder verzaubert werden. Sie
möchte immer weiter wachsen indem sie immer mehr Wasser aufnimmt. Wasser nähert
sie sich langsam, folgt aber der verlockung großer Wasseransammlungen.

\subsubsection{5 - Predigtenraum:} 2 in 6 Chance das Alwin (s.u., Raum 6) eine
Predigt abhält. Zuhörer der Predigt bestehen aus 1W6 normale Menschen mit 1W4-1
Gallertmenschen sowie 1W4 Werratten (in Menschenform) mit 1W4-2
Gallertmenschen.

Wird von dem Steinpodest an der Nordwand gesprochen muss jeder den Anweisungen
für 1W6 Tage folgen, wenn ein Rettungswurf gegen Zaubersprüche nicht geschafft
wurde.

Der Wandteppich im Süden Raumes verdeckt den Zugang zu Raum 7. Vor dem Podest
stehen drei alte, 8~m lange Holzbänke. Die Geheimtür im Nordwesten des Raumes kann
durch das leichtes Drücken geöffnet werden, wenn man sie entdeckt hat.

\textbf{Werratten} TW~3* RK~7~[12] (RK~9~[10]
in menschlicher Form) 1W6~(Biss) oder 1W6~(Waffe) K3 BW~12'' ML~8 EP~350
[4 in 6 Chance auf erfolgreichen Hinterhalt] 

\textbf{Normale Menschen} TW~1/2 RK~9~[10] 1W6~(Knüppel) NM BW~12'' ML~6 EP~50

\textbf{Gallertmenschen} TW~4 RK~7~[12] 2W6 K4 BW~12'' ML~6/12 EP~400

\subsubsection{6 - Bibliothek:} In den größtenteils leerstehenden Regalen
befinden sich 10 Tontafeln (Spruchrollen)[5 x Schutzrolle vor Bösem, Schild,
Unsichtbarkeit, Klopfen, Feuerball, Wachstum von Pflanzen]. In der
nordöstlichen Ecke befindet sich ein Schreibplatz, auf dem ein Schreibgriffel
liegt. In der nordwestlichen Ecke steht eine Statue der Artemis, hinter der ein
Geheimraum mit einer nach oben führenden Leiter ist. Vor der Statue befinden
sich Kratzspuren auf den Bodenplatten. 1 in 6 Chance das die Statue nicht vor
der Öffnung zum Geheimraum steht. 1 in 6 Chance das sich Alwin in dem Raum
befindet.

\textbf{Alwin [Medium]} TW~1 RK~9~[10]
1W6 (Dolch) oder Zauberspruch M1 BW~12'' ML~7 EP~100 [1 x Schutz vor Bösem]
 

\subsubsection{7 - Geheimtunnel:} In der Ecke in einer verschlossenen Truhe
befinden sich 400~GM. Dahinter, gegen die Wand gelehnt, steht Borstenschreck.
Wandteppiche grenzen den Geheimtunnel von Raum 4 und 5 ab. Drei Meter oberhalb
des nördlichen Ausgangs befindet sich ein Loch in der Wand (führt zu Gang G1).

\textbf{Borstenschreck:} magischer Speer +2, auf dem Schaft aus
Elfenbein sind Szenen einer Wildschweinjagd dargestellt.

\subsubsection{8 - Höhle mit Falltür:} Dieser Bereich ist leer. Am südlichen
Ausgang befindet sich eine, mit lockerer Erde verdeckte Falltür (Scharniere am
südlichen Ende). Wer die Falltür betritt fällt 3~m tief in ein 4~m tiefes
Wasserbecken. Schwere Rüstungen und hohe Traglast verhindern das Schwimmen und
ziehen zum Grund.  Die Falltür wird von den Werratten regelmäßig kontrolliert
und wieder geschlossen und verdeckt. Durch den Durchgang in der Wand kann ein
Halbling kriechen (oder Riesenratten).

\subsubsection{9 - Höhle mit Riesenschlagfalle:} An der westlichen Wand
befindet sich eine Anhäufung von Münzen (900~KM, 400~SM). Die Erde der Wand
hinter dem Haufen ist sehr glatt (durch den Bügel der Schlagfalle geglättet).
Die Riesenschlagfalle ist in den Boden eingelassen. Der Eisenbügel ist sichtbar
und nicht versteckt. Die Falle wird ausgelöst sobald Münzen von dem Haufen
genommen werden. Ein erfolgreicher Rettungswurf gegen Odem
verhindert, dass der
Abenteurer vom Bügel der Schlagfalle erschlagen wird (sofortiger Tod). Im
Norden ist die Wand zu Raum 4 durchbrochen.

\subsubsection{10 - Schlafplatz von Alwins Getreuen:} In diesem Bereich wohnen
alle die Alwin treu ergeben sind, egal ob aus freien Stücken oder beeinflusst
durch die Magie des Podestes aus Raum 5. Es sind 1W6 Werratten (in menschlicher
Form) und 1W2 normale Menschen anwesend. Jeder anwesende Mensch und jede
anwesende Werratte ist mit seinem Gallertmenschen dort. Ein Paar bewacht zwei
Diamanten (500~GM und 300~GM wert).

\textbf{Werratten} TW~3* RK~7~[12] (RK 9~[10] in menschlicher Form)
1W6~(Biss) oder 1W6~(Waffe) K3 BW~12'' ML~8 EP~350 [4 in 6 Chance
auf erfolgreichen Hinterhalt] 

\textbf{Normale Menschen} TW~1/2 RK~9~[10] 1W6~(Knüppel) NM BW~12'' ML~6 EP~50 

\textbf{Gallertmenschen} TW~4 RK~7~[12] 2W6 K4 BW~12'' ML~6/12 EP~400

\subsubsection{11 - Schlafplatz der autonomen Werratten:} Hier ruhen 1W4
Werratten in Riesenrattenform. Sie bewachen die Schätze der Werratten die weder
mit Alwin noch mit Balto zusammenarbeiten. Die Schätze sind in einem
Geheimraum, der durch einen Tunnel erreicht werden kann. Durch den Tunnel
passen Riesenratten, ein Halbling könnte ebenfalls durchkriechen. Eine
Diamanten besetzte, goldene Schwertlilienbrosche (500~GM) und ein magisches
Bronzekurzschwert sind hier versteckt.

\textbf{Werratten} TW~3* RK~7~[12] 1W6(Biss) oder 1W6(Waffe) K3 BW~12'' ML~8 EP~350
[4 in 6 Chance auf erfolgreichen Hinterhalt] 

\textbf{Magisches Bronzekurzschwert:} Kurzschwert+1, +2 gegen Lykanthropen

\subsubsection{12 - Schatzhöhle:} In den Wände befinden sich auf 1~m Höhe
insgesamt 40 kleine Öffnungen, die alle 0,5~m in die Wand gehen. In die
Öffnungen passt gerade so ein Arm hinein. Alle enthalten 1W10~EM und 1W20~GM.
Ein Abenteurer kann 10 Öffnungen in einer Phase leeren.

\subsubsection{G1 - Geheimer Kriechgang:} Der Gang ist so niedrig (60~cm) das selbst
Halblinge und Zwerge nicht aufrecht stehen können. Um in Raum 1 zu
gelangen darf die Artemisstatue den Durchgang nicht verdecken. Zu Raum
6 führt eine Leiter nach unten. Die Öffnung zu Raum 7 ist 3~m oberhalb
des Boden des Raumes.

\subsubsection{G2 - Baltos alte Experimentierhöhle:} Die Decke ist 1,5~m hoch. 4
Skeletthunde bewachen mit Gelbschimmel überzogene Rubine (50~GM, 18~GM,
113~GM, 76~GM, 89~GM, 5~GM, 29~GM, 47~GM, 112~GM, 63~GM) sowie einen ebenfalls
mit Gelbschimmel überzogenen Zauberstab der Lähmung [2 Ladungen]. Die
Gegenstände liegen an der westlichen Wand. Die Skeletthunde verlassen
den Raum nur wenn die Gegenstände gestohlen werden oder sie angegriffen
werden.

\textbf{Skeletthunde} TW~1 RK~7~[12] 1W6 K1 BW~12'' ML~12 EP~100 

\textbf{Gelbschimmel} TW~2 RK~- 1W6 + Ersticken K2 BW~0'' ML~12 EP~200 [Schaden
nur durch Feuer (1W4)]

Ersticken: Rettungswurf gegen Tod, Fehlschlag Tod in 6 Runden

\subsection{Gallertmenschen}

\begin{tabbing}
Anzahl: 1\\
Gesinnung: [siehe Beschreibung]\\
Bewegung: 12'' [siehe Beschreibung]\\
Rüstungsklasse: 7 [12]\\
Trefferwürfel: 4\\
Angriffe: 1\\
Schaden: 2W6\\
Rettungswurf: K4\\
Moral: 6/12 [siehe Beschreibung]\\
Hortklasse: keine\\
\end{tabbing}

Sie entsprechen sowohl den wahren als auch falschen Erinnerungen der
Personen, welche die gallertartige Masse im Zisternenbecken berühren
und überleben. Ihr Körper besteht aus einer gallertartigen Masse,
nur deren äußeres Erscheinungsbild, aber auch die Gesinnung und die
Bewegungsweite entsprechen den Erinnerungen. Wird
die Person, aus dessen Errinnerungen ein Gallertmensch geschaffen
wurde attackiert oder getötet, werden die Gallertmenschen wütend
und kämpfen bis zum Ende (dann Moral 12, sonst Moral 6).

Immun gegen Gift, Verzauberung, Lähmung und Versteinerung.

\by{olupo}

